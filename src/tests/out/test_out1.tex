

% *** Groupe de question groupe1 : 1 éléments***
% pour insérer une question tirée aléatoirement depuis ce groupe, utiliser \insertgroup[1]{groupe1}
\element{groupe1}{
    \begin{question}{quopen1}
    (2 pts) open question 1
    \AMCOpen{lines=3,dots=false,question=quopen1}{
        \wrongchoice[F]{f}
        \scoring{0}
        \wrongchoice[P]{p}
        \scoring{1}
        \correctchoice[J]{j}
        \scoring{2}
    }    
\end{question}

}

% fin groupe groupe1

% *** Groupe de question groupe2 : 1 éléments***
% pour insérer une question tirée aléatoirement depuis ce groupe, utiliser \insertgroup[1]{groupe2}
\element{groupe2}{
    \begin{question}{quopen2}
    (2 pts) open question 2
    \AMCOpen{lines=3,dots=false,question=quopen2}{
        \wrongchoice[F]{f}
        \scoring{0}
        \wrongchoice[P]{p}
        \scoring{1}
        \correctchoice[J]{j}
        \scoring{2}
    }    
\end{question}

}

% fin groupe groupe2

% *** Groupe de question groupe3 : 2 éléments***
% pour insérer une question tirée aléatoirement depuis ce groupe, utiliser \insertgroup[1]{groupe3}
\element{groupe3}{
    \begin{question}{groupe3-1}
    (1 pts) open question 3
    \AMCOpen{lines=1,dots=false,question=groupe3-1}{
        \wrongchoice[F]{f}
        \scoring{0}
        \wrongchoice[P]{p}
        \scoring{1}
        \correctchoice[J]{j}
        \scoring{2}
    }    
\end{question}

}
\element{groupe3}{
    \begin{question}{groupe3-2}
    (1 pts) open question 4
    \AMCOpen{lines=1,dots=false,question=groupe3-2}{
        \wrongchoice[F]{f}
        \scoring{0}
        \wrongchoice[P]{p}
        \scoring{1}
        \correctchoice[J]{j}
        \scoring{2}
    }    
\end{question}

}

% pour mélanger aléatoirement les questions du groupe de question
\setgroupmode{groupe3}{withreplacement}
% fin groupe groupe3

% *** Groupe de question questiongroup-1 : 1 éléments***
% pour insérer une question tirée aléatoirement depuis ce groupe, utiliser \insertgroup[1]{questiongroup-1}
\element{questiongroup-1}{
    \begin{question}{questiongroup-1}
    (1 pts) sipgnle choice quest 1
    \begin{multicols}{2}
        \begin{choices}
        \end{choices}
    \end{multicols}
\end{question}

}

% fin groupe questiongroup-1