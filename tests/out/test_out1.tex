\newsavebox{\ecvaqjgpnd}
\savebox{\ecvaqjgpnd}{\parbox[t]{14cm}{\color{red}{
une super réponse 1
}}}

% *** Groupe de question groupe1 : 1 éléments***
% pour insérer une question tirée aléatoirement depuis ce groupe, utiliser \insertgroup[1]{groupe1}
\element{groupe1}{
    \begin{question}{quopen1}
    (2 pts) open question 1
    \AMCOpen{lines=3,dots=false,question=quopen1,answer=\usebox{\ecvaqjgpnd}}{
        \wrongchoice[F]{f}
        \scoring{0}
        \wrongchoice[P]{p}
        \scoring{1}
        \correctchoice[J]{j}
        \scoring{2}
    }    
\end{question}

}

% fin groupe groupe1\newsavebox{\ttntplhida}
\savebox{\ttntplhida}{\parbox[t]{14cm}{\color{red}{
une autre super réponse
}}}

% *** Groupe de question groupe2 : 1 éléments***
% pour insérer une question tirée aléatoirement depuis ce groupe, utiliser \insertgroup[1]{groupe2}
\element{groupe2}{
    \begin{question}{quopen2}
    (2 pts) open question 2
    \AMCOpen{lines=3,dots=false,question=quopen2,answer=\usebox{\ttntplhida}}{
        \wrongchoice[F]{f}
        \scoring{0}
        \wrongchoice[P]{p}
        \scoring{1}
        \correctchoice[J]{j}
        \scoring{2}
    }    
\end{question}

}

% fin groupe groupe2
% *** Groupe de question groupe3 : 2 éléments***
% pour insérer une question tirée aléatoirement depuis ce groupe, utiliser \insertgroup[1]{groupe3}
\element{groupe3}{
    \begin{question}{groupe3-1}
    (1 pts) open question 3
    \AMCOpen{lines=1,dots=false,question=groupe3-1}{
        \wrongchoice[F]{f}
        \scoring{0}
        \wrongchoice[P]{p}
        \scoring{1}
        \correctchoice[J]{j}
        \scoring{2}
    }    
\end{question}

}
\element{groupe3}{
    \begin{question}{groupe3-2}
    (1 pts) open question 4
    \AMCOpen{lines=1,dots=false,question=groupe3-2}{
        \wrongchoice[F]{f}
        \scoring{0}
        \wrongchoice[P]{p}
        \scoring{1}
        \correctchoice[J]{j}
        \scoring{2}
    }    
\end{question}

}

% pour mélanger aléatoirement les questions du groupe de question
\setgroupmode{groupe3}{withreplacement}
% fin groupe groupe3
% *** Groupe de question questiongroup-1 : 1 éléments***
% pour insérer une question tirée aléatoirement depuis ce groupe, utiliser \insertgroup[1]{questiongroup-1}
\element{questiongroup-1}{
    \begin{question}{questiongroup-1}
    (1 pts) sipgnle choice quest 1
    \begin{multicols}{2}
        \begin{choices}
        \end{choices}
    \end{multicols}
\end{question}

}

% fin groupe questiongroup-1\newsavebox{\psljpqbfwx}
\savebox{\psljpqbfwx}{\parbox[t]{14cm}{\color{red}{
tantant
}}}

% *** Groupe de question questiongroup-2 : 1 éléments***
% pour insérer une question tirée aléatoirement depuis ce groupe, utiliser \insertgroup[1]{questiongroup-2}
\element{questiongroup-2}{
    \begin{question}{questiongroup-2}
    (3 pts) titi toutou
    \AMCOpen{lines=2,dots=false,question=questiongroup-2,answer=\usebox{\psljpqbfwx}}{
        \wrongchoice[F]{f}
        \scoring{0}
        \wrongchoice[P]{p}
        \scoring{1}
        \correctchoice[J]{j}
        \scoring{2}
    }    
\end{question}

}

% fin groupe questiongroup-2