
\element{groupetest}{
	\begin{question}{subnet1-0}\sanbar{1}
On donne l'adresse suivante en notation CIDR: 62.21.67.188/30, quelle est l'adresse du sous-réseau?
		\begin{multicols}{3}
            \begin{choices}
				\wrongchoice{62.21.67.188}
				\correctchoice{62.21.67.188}
				\wrongchoice{255.255.255.252}
				\wrongchoice{62.21.67.255}
				\wrongchoice{62.21.67.30}
				\wrongchoice{62.21.67.189}
			\end{choices}
        \end{multicols}
	\end{question}
}
\element{groupetest}{
	\begin{question}{subnet1-1}\sanbar{1}
On donne l'adresse suivante en notation CIDR: 189.244.129.161/29, quelle est l'adresse du sous-réseau?
		\begin{multicols}{3}
            \begin{choices}
				\wrongchoice{189.244.129.161}
				\correctchoice{189.244.129.160}
				\wrongchoice{255.255.255.248}
				\wrongchoice{189.244.129.255}
				\wrongchoice{189.244.129.29}
				\wrongchoice{189.244.129.161}
			\end{choices}
        \end{multicols}
	\end{question}
}





\element{donneestables1}{
\begin{question}{q-dt1}\sanbar{2}
    On a extrait les deux premières lignes de différents fichiers.
    Déterminer celui qui est un authentique fichier CSV :
    \begin{choices}
        \correctchoice{Nom,Pays,Temps

        Camille Muffat,France,241.45}
        \wrongchoice{Nom Pays Temps
        
        Camille Muffat France 241.45}
        \wrongchoice{[
        
        \{ "Nom": "Camille Muffat", "Pays": "France", "Temps": 241.45\},}
        \wrongchoice{[
        
        \{ Nom: "Camille Muffat", Pays: "France", Temps: 241.45\},}
    \end{choices}
\end{question}
}


\element{donneestables1}{  
\begin{question}{q-dt3}\sanbar{2} On donne le tableau de `mendeleiev` construit de cette manière:
    
    mendeleiev = [['H','.','.','.','.','.','.','He'],

    ['Li','Be','B','C','N','O','Fl','Ne'],

    ['Na','Mg','Al','Si','P','S','Cl','Ar']]

    Comment produire la liste des gaz rares, c'est-à-dire la liste des éléments de la dernière colonne ?
    \begin{choices}
        \correctchoice{gaz\_rares = [periode[7] for periode in mendeleiev]}
        \wrongchoice{gaz\_rares = [periode for periode in mendeleiev[7]]}
        \wrongchoice{gaz\_rares = [periode for periode[7] in mendeleiev]}
        \wrongchoice{gaz\_rares = [periode[8] for periode in mendeleiev]}
    \end{choices}
\end{question}
}
\element{donneestables1}{
\begin{question}{q7}\sanbar{2}
    On exécute le code suivant :
    \inputminted{python}{./prog4.py}
    Quelle est la valeur de la variable v à la fin de cette exécution ?
    \begin{multicols}{2}
        \begin{choices}
            \correctchoice{[3,6,9]}
            \wrongchoice{18}
            \wrongchoice{[1,4,7]}
            \wrongchoice{[1,2,3,4,5,6,7,8,9]}
        \end{choices}
    \end{multicols}
\end{question}
}
\setgroupmode{donneestables1}{withoutreplacement}







\element{programmation1}{
	\begin{question}{q4}\sanbar{2}
	Quelle est la valeur affichée à l'exécution du programme Python suivant ?
	\inputminted{python}{./prog1.py}
	\begin{multicols}{3}
		\begin{choices}
			\wrongchoice{3}
			\correctchoice{21}
			\wrongchoice{\{ 'ftp': 21 \}}
			\wrongchoice{Key not found}
		\end{choices}
	\end{multicols}
\end{question}
}
\element{programmation1}{
\begin{question}{q5}\sanbar{2}
	On exécute le script suivant:
	\inputminted{python}{./prog2.py}
	Quelle est la valeur de m à la fin de son exécution ?
	\begin{choices}
		\wrongchoice{[[0, 0, 0, 0, 0], [0, 1, 2, 3, 4], [0, 2, 4, 6, 8]]}
		\correctchoice{[[0, 0, 0], [0, 1, 2], [0, 2, 4], [0, 3, 6], [0, 4, 8]]}
		\wrongchoice{[[1, 1, 1], [2, 4, 6], [3, 6, 9], [4, 8, 12], [5, 10, 15]]}
		\wrongchoice{[[1, 1, 1, 1, 1], [2, 4, 6, 8, 10], [3, 6, 9, 12, 15], [4, 8, 12, 16, 20], [5, 10,15, 20, 25]]}
	\end{choices}
\end{question}
}
\element{programmation1}{
\begin{question}{q6}\sanbar{2}
	À quelle affectation sont équivalentes les instructions suivantes, où a, b sont des variables entières et c une
variable booléenne ?
	\inputminted{python}{./prog3.py}
	Quelle est la valeur de m à la fin de son exécution ?
	\begin{multicols}{2}
		\begin{choices}
			\correctchoice{c = (a==b) or (a > b+10)}
			\wrongchoice{c = (a==b) and (a > b+10)}
			\wrongchoice{c = not(a==b)}
			\wrongchoice{c = not(a > b+10)}
		\end{choices}
	\end{multicols}
\end{question}
}

\setgroupmode{programmation1}{withoutreplacement}
